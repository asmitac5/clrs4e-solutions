Let's assume that the base case is $T(n)=\Theta(1)$ for $n<2$ and let $c$ represent the upper-bound constant hidden by the $\Theta(\lg n)$ term for $n\ge2$.
\refFigure{4-3} shows the recursion tree for the recurrence $T(n)=2T\bigl(\sqrt{n}\bigr)+c\lg n$.
\begin{figure}[htb]
    \subimport{}{main.tikz}
\begin{tikzpicture}
    \node (root1) {$x$}
        child[missing]
        child {node {$y$}
            child[missing]
            child {node {$z$}}};
    \node[right=3\vertexsize of root1] (root2) {$x$}
        child[missing]
        child {node {$y$}
            child {node {$z$}}
            child[missing]};
    \node[right=4\vertexsize of root2] (root3) {$x$}
        child {node {$y$}}
        child {node {$z$}};
    \node[right=4\vertexsize of root3] (root4) {$x$}
        child {node {$y$}
            child[missing]
            child {node {$z$}}}
            child[missing];
    \node[right=3\vertexsize of root4] (root5) {$x$}
        child {node {$y$}
            child {node {$z$}}
            child[missing]}
        child[missing];
    \node[below=5\vertexsize of root1] (root6) {$x$}
        child[missing]
        child {node {$z$}
            child[missing]
            child {node {$y$}}};
    \node[right=3\vertexsize of root6] (root7) {$x$}
        child[missing]
        child {node {$z$}
            child {node {$y$}}
        child[missing]};
    \node[right=4\vertexsize of root7] (root8) {$x$}
        child {node {$z$}}
        child {node {$y$}};
    \node[right=4\vertexsize of root8] (root9) {$x$}
        child {node {$z$}
            child[missing]
            child {node {$y$}}}
        child[missing];
    \node[right=3\vertexsize of root9] {$x$}
        child {node {$z$}
            child {node {$y$}}
            child[missing]}
        child[missing];
\end{tikzpicture}

    \caption{A recursion tree for the recurrence $T(n)=2T\bigl(\sqrt{n}\bigr)+c\lg n$.} \label{fig:4-3}
\end{figure}

Let $h$ be the height of the tree.
The cost of the root is $c\lg n$.
The recurrence breaks down a problem of size $n$ into two subproblems, each of size $\sqrt{n}$, and each incurring the cost of $c\lg\sqrt{n}=c\lg n^{1/2}=(c\lg n)/2$.
In general, at depth $i$, where $i=0$, 1, \dots, $h-1$, there are $2^i$ subproblems, each of size $n^{1/2^i}$\!, and the total level cost is $2^ic\lg\bigl(n^{1/2^i}\!\bigr)=(2^ic\lg n)/2^i=c\lg n$.
The bottom level contains leaves representing subproblems of size less than 2, so we have $n^{1/2^{h-1}}\!\ge2$ and $n^{1/2^h}\!<2$.
From there we get $\lg\lg n<h\le\lg\lg n+1$, or $h=\lfloor\lg\lg n\rfloor+1$ (by inequality (3.2)), leading to a total leaf cost of $2^h\cdot\Theta(1)=\Theta(\lg n)$.

Adding up the costs of all internal nodes and the costs of all leaves yields
\begin{align*}
    T(n) &= \sum_{i=0}^{h-1}c\lg n+\Theta(\lg n) \\
    &= c\lg n\cdot h+\Theta(\lg n) \\
    &= c\lg n(\lfloor\lg\lg n\rfloor+1)+\Theta(\lg n) \\
    &\le c\lg n(\lg\lg n+1)+\Theta(\lg n) \\
    &= O(\lg n\lg\lg n).
\end{align*}

We follow a similar reasoning for $c$ representing the lower-bound constant hidden by the $\Theta(\lg n)$ term in recurrence (4.25), to obtain $T(n)=\Omega(\lg n\lg\lg n)$.
