Our first guess is that $T(n)\le cn^2$ for all $n\ge n_0$, where $c$, $n_0>0$ are constants.
Letting $n\ge2n_0$ and substituting the inductive hypothesis applied to $T(n/2)$, yields
\begin{align*}
    T(n) &\le 4c(n/2)^2+n \\
    &= cn^2+n,
\end{align*}
but that does not imply that $T(n)\le cn^2$ for any choice of $c$.

Let's then improve our guess by subtracting a lower-order term: $T(n)\le cn^2-dn$, where $d\ge0$ is another constant.
Assume by induction that the bound holds for all values at least as big as $n_0$, but less than $n$.
For $n\ge2n_0$ we have $T(n/2)\le c(n/2)^2-d(n/2)$, and so
\begin{align*}
    T(n) &\le 4(c(n/2)^2-d(n/2))+n \\
    &= cn^2-2dn+n \\
    &= cn^2-dn-n(d-1) \\
    &\le cn^2-dn,
\end{align*}
where the last step holds as long as $d\ge1$.

Now let $n_0\le n<2n_0$.
Let's pick $n_0=1$.
Choosing $c=T(1)+d$ satisfies the condition $T(1)\le c\cdot1^2-d\cdot1$, handling the base case and completing the proof.
okie