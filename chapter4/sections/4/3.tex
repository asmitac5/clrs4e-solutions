We'll use the inductive hypothesis that $T(n)\ge dn\lg n$ for all $n\ge n_0$, where $d$, $n_0>0$ are some constants.
Assume that the inductive hypothesis holds for all numbers at least as big as $n_0$ and less than $n$.
Let $n\ge3n_0$.
Then, $2n/3\ge n/3\ge n_0$, so both $T(n/3)\ge d(n/3)\lg(n/3)$ and $T(2n/3)\ge d(2n/3)\lg(2n/3)$ hold.
Substituting into the recurrence yields
\begin{align*}
    T(n) &\ge d(n/3)\lg(n/3)+d(2n/3)\lg(2n/3)+\Theta(n) \\
    &= d(n/3)(\lg n-\lg3)+2d(n/3)(\lg n+\lg2-\lg3)+\Theta(n) \\
    &= d(n/3)(3\lg n+2-3\lg3)+\Theta(n) \\
    &= dn\lg n-(\lg3-2/3)dn+\Theta(n) \\
    &\ge dn\lg n.
\end{align*}
The last step holds if we constrain the constant $d$ to be sufficiently small that for $n\ge3n_0$, the anonymous function hidden by the $\Theta(n)$ term dominates the quantity $(\lg3-2/3)dn$.

Let's pick $n_0=2$, so that $\lg n>0$ for all $n\ge n_0$.
We can decrease $d$ enough, so to satisfy both $T(2)\ge d(2\lg2)$ and $T(3)\ge d(3\lg3)$, establishing the inductive hypothesis for the base cases.

We've shown that $T(n)\ge dn\lg n$ for all $n\ge2$, so $T(n)=\Omega(n\lg n)$.
Combining this result with the upper bound shown in the book, we get $T(n)=\Theta(n\lg n)$.
