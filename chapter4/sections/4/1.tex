\refFigure{4.4-1} shows the recursion trees for the recurrences solved in this exercise.
\vspace*{2\baselineskip}

\subexercise
For simplicity, assume that $n$ is an exact power of 2 and that the base case is $T(1)=\Theta(1)$.
At each level, the problem is reduced to a subproblem of exactly half the size, until the problem size hits $n=1$, so the height of the recursion tree is $\lg n$.
Each level of the tree has a single node which, at depth $i=0$, 1, \dots, $\lg n-1$, incurs the cost of $(n/2^i)^3=n^3\!/8^i$, and the leaf at depth $\lg n$ incurs the cost of $\Theta(1)$.
Therefore, the cost of the whole tree is
\begin{align*}
    T(n) &= \sum_{i=0}^{\lg n-1}n^3\!/8^i+\Theta(1) \\
    &< \sum_{i=0}^\infty n^3\!/8^i+\Theta(1) \\
    &= \frac{1}{1-(1/8)}\,n^3+\Theta(1) && \text{(by equation (A.7))} \\[1mm]
    &= (8/7)n^3+\Theta(1) \\
    &= O(n^3).
\end{align*}

To prove this result, we'll use the substitution method.
Let's adopt the inductive hypothesis that $T(n)\le cn^3$ for all $n\ge n_0$, where $c$, $n_0>0$ are constants, and assume that the inductive hypothesis holds for all values at least as big as $n_0$ and less than $n$.
Let $n\ge2n_0$, so that $T(n/2)\le c(n/2)^3$.
Substituting into the recurrence yields
\begin{align*}
    T(n) &\le c(n/2)^3+n^3 \\
    &= cn^3\!/8+n^3 \\
    &= (c/8+1)n^3 \\
    &\le cn^3,
\end{align*}
where the last step holds as long as $c/8+1\le c$, or $c\ge8/7$.

Let's pick $n_0=1$.
Picking $c=\max{8/7,T(1)}$ satisfies $T(1)\le c\cdot1^3$, establishing the inductive hypothesis for the base case.

We've shown that $T(n)\le cn^3$ for all $n\ge1$, which implies that the solution to the recurrence is $T(n)=O(n^3)$.

\subexercise
Assume that $n$ is an exact power of 3 and that $T(1)=\Theta(1)$.
The subproblem size for a node at depth $i$ is $n/3^i$, and the bottom level of the tree is at depth $i$ such that $n/3^i=1$ or, equivalently, $i=\log_3n$.
Each level below the top has four times as many nodes as the level above, and so the number of nodes at depth $i$ is $4^i$.
The total cost of all nodes at a given depth $i$ is $4^i(n/3^i)=(4/3)^in$.
The bottom level contains $4^{\log_3n}=n^{\log_34}$ leaves, each contributing $\Theta(1)$, leading to a total leaf cost of $\Theta(n^{\log_34})$.
Therefore, the cost of the entire tree is
\begin{align*}
    T(n) &= \sum_{i=0}^{\log_3n-1}(4/3)^in+\Theta(n^{\log_34}) \\[1mm]
    &= \frac{(4/3)^{\log_3n}-1}{4/3-1}\,n+\Theta(n^{\log_34}) && \text{(by equation (A.6))} \\[1mm]
    &< 3n(4/3)^{\log_3n}+\Theta(n^{\log_34}) \\[1mm]
    &= 3n\,\frac{n^{\log_34}}{n^{\log_33}}+\Theta(n^{\log_34}) && \text{(by equation (3.21))} \\[2mm]
    &= 3n^{\log_34}+\Theta(n^{\log_34}) \\
    &= O(n^{\log_34}).
\end{align*}

To verify this bound using the substitution method, we adopt the inductive hypothesis that $T(n)\le cn^{\log_34}-dn$ for all $n\ge n_0$, where $c$, $n_0>0$ and $d\ge0$ are some constants.
Assume that the inductive hypothesis holds for all numbers at least as big as $n_0$ and less than $n$.
If we let $n\ge3n_0$, we have $T(n/3)\le c(n/3)^{\log_34}-d(n/3)$.
Substituting into the recurrence yields
\begin{align*}
    T(n) &\le 4\left(c(n/3)^{\log_34}-d(n/3)\right)+n \\
    &= 4cn^{\log_34}\!/3^{\log_34}-4dn/3+n \\
    &= 4cn^{\log_34}\!/4^{\log_33}-4dn/3+n \\
    &= cn^{\log_34}-dn+n(1-d/3) \\
    &\le cn^{\log_34}-dn,
\end{align*}
where the last step holds as long as $1-d/3\le0$, or $d\ge3$.

Let's pick $n_0=1$.
Picking $c=\max{T(1)+d,(T(2)+2d)/2^{\log_34}}$ satisfies both conditions $T(1)\le c\cdot1^{\log_34}-d\cdot1$ and $T(2)\le c\cdot2^{\log_34}-d\cdot2$, establishing the inductive hypothesis for the base cases.

We've shown that $T(n)\le cn^{\log_34}-dn\le cn^{\log_34}$ for all $n\ge1$, which implies that the solution to the recurrence is $T(n)=O(n^{\log_34})$.

\subexercise
Assume that $n$ is an exact power of 2 and that $T(1)=\Theta(1)$.
The height of the recursion tree is $\lg n$.
At depth $i$, for $i=0$, 1, \dots, $\lg n-1$, there are $4^i$ nodes, each contributing $n/2^i$, so the total cost of all nodes at depth $i$ is $4^i(n/2^i)=2^in$.
The bottom level, at depth $\lg n$, contains $4^{\lg n}=n^2$ leaves, so the total leaf cost is $\Theta(n^2)$.
Thus, we get
\begin{align*}
    T(n) &= \sum_{i=0}^{\lg n-1}2^in+\Theta(n^2) \\
    &= \frac{2^{\lg n}-1}{2-1}\,n+\Theta(n^2) && \text{(by equation (A.6))} \\[1mm]
    &< 2^{\lg n}n+\Theta(n^2) \\
    &= n^2+\Theta(n^2) \\
    &= O(n^2).
\end{align*}

To formally prove this bound we adopt the inductive hypothesis that $T(n)\le cn^2-dn$ for all $n\ge n_0$, where $c$, $n_0>0$ and $d\ge0$ are constants, and assume that the inductive hypothesis holds for all numbers at least as big as $n_0$ and less than $n$.
When $n\ge2n_0$ it holds that $T(n/2)\le c(n/2)^2-d(n/2)$, so
\begin{align*}
    T(n) &\le 4(c(n/2)^2-d(n/2))+n \\
    &= cn^2-2dn+n \\
    &= cn^2-dn+n(1-d) \\
    &\le cn^2-dn && \text{(as long as $d\ge1$)}.
\end{align*}

If we now let $n_0=1$, we can choose $c=T(1)+d$ to satisfy $T(1)\le c\cdot1^2-d\cdot1$, establishing the bound for the base case.

We've shown that $T(n)\le cn^2-dn\le cn^2$ for all $n\ge1$, which implies that the solution to the recurrence is $T(n)=O(n^2)$.

\subexercise
Assume that $T(1)=\Theta(1)$.
The height of the recursion tree is $n$.
At depth $i$, for $i=0$, 1, \dots, $n-1$, there are $3^i$ nodes incurring a unit cost each, so the total cost of all nodes at depth $i$ is $3^i$.
The bottom level, at depth $n$, contains $3^n$ leaves, so the total leaf cost is $\Theta(3^n)$.
Thus, we get
\begin{align*}
    T(n) &= \sum_{i=0}^{n-1}3^i+\Theta(3^n) \\[1mm]
    &= \frac{3^n-1}{3-1}+\Theta(3^n) && \text{(by equation (A.6))} \\[1mm]
    &< 2\cdot3^n+\Theta(3^n) \\
    &= O(3^n).
\end{align*}

To prove this bound we adopt the inductive hypothesis that $T(n)\le c3^n-d$ for all $n\ge n_0$, where $c$, $n_0>0$ and $d\ge0$ are constants, and assume that the inductive hypothesis holds for all numbers at least as big as $n_0$ and less than $n$.
When $n\ge n_0+1$ it holds that $T(n-1)\le c3^{n-1}-d$, so
\begin{align*}
    T(n) &\le 3(c3^{n-1}-d)+1 \\
    &= c3^n-3d+1 \\
    &= c3^n-d+(1-2d) \\
    &\le c3^n-d && \text{(as long as $d\ge1/2$)}.
\end{align*}

If we now let $n_0=1$, we can choose $c=(T(1)+d)/3$ to satisfy $T(1)\le c\cdot3^1-d$, establishing the bound for the base case.

We've shown that $T(n)\le c3^n-d\le c3^n$ for all $n\ge1$, which implies that the solution to the recurrence is $T(n)=O(3^n)$.

\begin{figure}[!htb]
    \subcaptionbox{\label{fig:4.4-1a}}[\textwidth]{\begin{tikzpicture}

\def\circleA{(0:0) circle (7mm) node[index, above left=1mm and 2mm] (A) {$A$}}
\def\circleB{(0:5mm) circle (7mm) node[index, above right=1mm and 2mm] (B) {$B$}}
\def\circleC{(-60:5mm) circle (7mm) node[index, below=3mm] (C) {$C$}}

\begin{scope}
    \draw[fill=lightblue] \circleA;
    \draw \circleB;
    \draw \circleC +(0, -12mm) node[index] (label 1) {$A$};
\end{scope}

\node (center 1) at (barycentric cs:A=1,B=1,C=1) {};

\begin{scope}[xshift=28mm]
    \draw[fill=lightblue] \circleB \circleC +(0, -12mm) node[index] (label 2) {$(B\cup C)$};
    \draw \circleA;
\end{scope}

\node (center 2) at (barycentric cs:A=1,B=1,C=1) {};

\begin{scope}[xshift=56mm]
    \begin{scope}\clip \circleB;
        \draw[fill=lightblue] \circleA;
    \end{scope}
    \begin{scope}
        \clip \circleC;
        \draw[fill=lightblue] \circleA;
    \end{scope}
    \draw \circleA \circleB \circleC +(0, -12mm) node[index] (label 3) {$A\cap(B\cup C)$};
\end{scope}

\node (center 3) at (barycentric cs:A=1,B=1,C=1) {};

\begin{scope}[xshift=84mm]
    \begin{scope}
        \clip \circleB;
        \draw[fill=lightblue] \circleA;
    \end{scope}
    \draw \circleA \circleB \circleC +(0, -12mm) node[index] (label 4) {$(A\cap B)$};
\end{scope}

\node (center 4) at (barycentric cs:A=1,B=1,C=1) {};

\begin{scope}[xshift=112mm]
    \begin{scope}
        \clip \circleC;
        \draw[fill=lightblue] \circleA;
    \end{scope}
    \draw \circleA \circleB \circleC +(0, -12mm) node[index] (label 5) {$(A\cap C)$};
\end{scope}

\node (center 5) at (barycentric cs:A=1,B=1,C=1) {};

\path (center 1) -- (center 2) node[index, midway] (midlabel 1) {$\cap$}
      (center 2) -- (center 3) node[index, midway] (midlabel 2) {$=$}
      (center 3) -- (center 4) node[index, midway] (midlabel 3) {$=$}
      (center 4) -- (center 5) node[index, midway] (midlabel 4) {$\cup$};

\path (label 1) -| (midlabel 1) node[index, midway] {$\cap$}
      (label 2) -| (midlabel 2) node[index, midway] {$=$}
      (label 3) -| (midlabel 3) node[index, midway] {$=$}
      (label 4) -| (midlabel 4) node[index, midway] {$\cup$};

\end{tikzpicture}

\begin{tikzpicture}[total node/.append style={text width=12mm}]

\node (root) {$n^3$}
	child {node (0) {$\bigl(\frac{n}{2}\bigr)^3$}
		child {node (00) {$\bigl(\frac{n}{4}\bigr)^3$}
			child {node (000) {}}
		}
	};

\node[below=5mm of 000] (leaf) {$\Theta(1)$};
\draw[transition edge={draw=none}{very densely dashed}{0.5}] (000) -- (leaf);

\node[total node, right=of leaf] (level h total) {$\Theta(1)$};
\node[total node] at (level h total |- 00) (level 2 total) {$\bigl(\frac{1}{8}\bigr)^2n^3$};
\node[total node] at (level h total |- 0) (level 1 total) {$\frac{1}{8}\,n^3$};
\node[total node] at (level h total |- root) (level 0 total) {$n^3$};

\path (level 2 total.east) -- (level h total.east) node[midway, xshift=-5mm, font=\bfseries] {$\vdots$};

\draw[level arrow, shorten <=2mm] (root) -- (level 0 total);
\draw[level arrow, shorten <=2mm] (0) -- (level 1 total);
\draw[level arrow, shorten <=2mm] (00) -- (level 2 total);
\draw[level arrow, shorten <=2mm] (leaf) -- (level h total);

\coordinate[below=5mm of leaf] (leaves total);
\coordinate (total line end) at (leaves total -| level h total.east);
\draw (0 |- total line end) -- (total line end) node[at end, anchor=east, yshift=-4mm] {Total: $O(n^3)$};

\coordinate[left=4mm of leaf] (bottom);
\draw[<->, orange] (bottom) -- (bottom |- root) node[midway, fill=white, text=black] {$\lg n$};

\end{tikzpicture}
}
    \par\vspace{10mm}
    \subcaptionbox{\label{fig:4.4-1b}}[\textwidth]{\begin{tikzpicture}

\def\circleA{(0:0) circle (7mm) node[index, above left=1mm and 2mm] (A) {$A$}}
\def\circleB{(0:5mm) circle (7mm) node[index, above right=1mm and 2mm] (B) {$B$}}
\def\circleC{(-60:5mm) circle (7mm) node[index, below=3mm] (C) {$C$}}

\begin{scope}
    \draw[fill=lightblue] \circleA;
    \draw \circleB;
    \draw \circleC +(0, -12mm) node[index] (label 1) {$A$};
\end{scope}

\node (center 1) at (barycentric cs:A=1,B=1,C=1) {};

\begin{scope}[xshift=28mm]
    \draw[fill=lightblue] \circleB \circleC +(0, -12mm) node[index] (label 2) {$(B\cup C)$};
    \draw \circleA;
\end{scope}

\node (center 2) at (barycentric cs:A=1,B=1,C=1) {};

\begin{scope}[xshift=56mm]
    \begin{scope}\clip \circleB;
        \draw[fill=lightblue] \circleA;
    \end{scope}
    \begin{scope}
        \clip \circleC;
        \draw[fill=lightblue] \circleA;
    \end{scope}
    \draw \circleA \circleB \circleC +(0, -12mm) node[index] (label 3) {$A\cap(B\cup C)$};
\end{scope}

\node (center 3) at (barycentric cs:A=1,B=1,C=1) {};

\begin{scope}[xshift=84mm]
    \begin{scope}
        \clip \circleB;
        \draw[fill=lightblue] \circleA;
    \end{scope}
    \draw \circleA \circleB \circleC +(0, -12mm) node[index] (label 4) {$(A\cap B)$};
\end{scope}

\node (center 4) at (barycentric cs:A=1,B=1,C=1) {};

\begin{scope}[xshift=112mm]
    \begin{scope}
        \clip \circleC;
        \draw[fill=lightblue] \circleA;
    \end{scope}
    \draw \circleA \circleB \circleC +(0, -12mm) node[index] (label 5) {$(A\cap C)$};
\end{scope}

\node (center 5) at (barycentric cs:A=1,B=1,C=1) {};

\path (center 1) -- (center 2) node[index, midway] (midlabel 1) {$\cap$}
      (center 2) -- (center 3) node[index, midway] (midlabel 2) {$=$}
      (center 3) -- (center 4) node[index, midway] (midlabel 3) {$=$}
      (center 4) -- (center 5) node[index, midway] (midlabel 4) {$\cup$};

\path (label 1) -| (midlabel 1) node[index, midway] {$\cap$}
      (label 2) -| (midlabel 2) node[index, midway] {$=$}
      (label 3) -| (midlabel 3) node[index, midway] {$=$}
      (label 4) -| (midlabel 4) node[index, midway] {$\cup$};

\end{tikzpicture}

\begin{tikzpicture}[
	level/.append style = {sibling distance=18\vertexsize/4^#1},
	total node/.append style={text width=15mm}
]

\node (root) {$n$}
	child {node (0) {$\frac{n}{3}$}
		child {node (00) {$\frac{n}{9}$}
			child {node (000) {}}
			child {node {}}
			child {node {}}
			child {node {}}
		}
		child {node (01) {$\frac{n}{9}$}
			child {node {}}
			child {node {}}
			child {node {}}
			child {node {}}	
		}
		child {node (02) {$\frac{n}{9}$}
			child {node {}}
			child {node {}}
			child {node {}}
			child {node {}}
		}
		child {node (03) {$\frac{n}{9}$}
			child {node {}}
			child {node {}}
			child {node {}}
			child {node {}}
		}
	}
	child {node (1) {$\frac{n}{3}$}
		child {node (10) {$\frac{n}{9}$}
			child {node {}}
			child {node {}}
			child {node {}}
			child {node {}}
		}
		child {node (11) {$\frac{n}{9}$}
			child {node {}}
			child {node {}}
			child {node {}}
			child {node {}}
		}
		child {node (12) {$\frac{n}{9}$}
			child {node {}}
			child {node {}}
			child {node {}}
			child {node {}}
		}
		child {node (13) {$\frac{n}{9}$}
			child {node {}}
			child {node {}}
			child {node {}}
			child {node {}}
		}
	}
	child {node (2) {$\frac{n}{3}$}
		child {node (20) {$\frac{n}{9}$}
			child {node {}}
			child {node {}}
			child {node {}}
			child {node {}}
		}
		child {node (21) {$\frac{n}{9}$}
			child {node {}}
			child {node {}}
			child {node {}}
			child {node {}}
		}
		child {node (22) {$\frac{n}{9}$}
			child {node {}}
			child {node {}}
			child {node {}}
			child {node {}}
		}
		child {node (23) {$\frac{n}{9}$}
			child {node {}}
			child {node {}}
			child {node {}}
			child {node {}}
		}
	}
	child {node (3) {$\frac{n}{3}$}
		child {node (30) {$\frac{n}{9}$}
			child {node {}}
			child {node {}}
			child {node {}}
			child {node {}}
		}
		child {node (31) {$\frac{n}{9}$}
			child {node {}}
			child {node {}}
			child {node {}}
			child {node {}}
		}
		child {node (32) {$\frac{n}{9}$}
			child {node {}}
			child {node {}}
			child {node {}}
			child {node {}}
		}
		child {node (33) {$\frac{n}{9}$}
			child {node {}}
			child {node {}}
			child {node {}}
			child {node (333) {}}
		}
	};
	
\node[below=5mm of 000] (leaf 0) {$\Theta(1)$};
\foreach[count=\i] \x in {0, ..., 5} {
	\node[right=0pt of leaf \x] (leaf \i) {$\Theta(1)$};
}
\node[below=5mm of 333] (leaf m) {$\Theta(1)$};
\node[left=0pt of leaf m] (leaf m minus 1) {$\Theta(1)$};
\node[left=0pt of leaf m minus 1] (leaf m minus 2) {$\Theta(1)$};
\path (leaf 6) -- (leaf m minus 2) node[midway, font=\bfseries] {$\dots$};
\foreach \x in {0, ..., 6, m minus 2, m minus 1, m} {
	\draw[transition edge={draw=none}{very densely dashed}{0.5}] (leaf \x |- 000) -- (leaf \x.north);
}

\node[total node, right=of leaf m] (level h total) {$\Theta(n^{\log_34})$};
\node[total node] at (level h total |- 22) (level 2 total) {$\bigl(\frac{4}{3}\bigr)^2n$};
\node[total node] at (level h total |- 2) (level 1 total) {$\frac{4}{3}\,n$};
\node[total node] at (level h total |- root) (level 0 total) {$n$};

\path (level 2 total.east) -- (level h total.east) node[midway, xshift=-5mm, font=\bfseries] {$\vdots$};

\draw[level arrow, shorten <=5mm] (root) -- (level 0 total);
\draw[level arrow, shorten <=5mm] (3) -- (level 1 total);
\draw[level arrow, shorten <=5mm] (33) -- (level 2 total);
\draw[level arrow, shorten <=2mm] (leaf m) -- (level h total);

\draw[orange, decorate, decoration={brace, amplitude=8pt, mirror}]
	(leaf 0.south west) -- (leaf m.south east) node[midway, yshift=-5mm, text=black] (leaves total) {$4^{\log_3n}=n^{\log_34}$};

\coordinate (total line end) at (leaves total -| level h total.east);
\draw (3 |- total line end) -- (total line end) node[at end, anchor=east, yshift=-4mm] {Total: $O(n^{\log_34})$};

\coordinate[left=4mm of leaf 0] (bottom);
\draw[<->, orange] (bottom) -- (bottom |- root) node[midway, fill=white, text=black] {$\log_3n$};

\end{tikzpicture}
}
    \caption{
        \textbf{(a)}\, A recursion tree for the recurrence $T(n)=T(n/2)+n^3$.\,
        \textbf{(b)}\, A recursion tree for the recurrence $T(n)=4T(n/3)+n$.
    } \label{fig:4.4-1}
\end{figure}

\begin{figure}[htb!]\ContinuedFloat
    \subcaptionbox{\label{fig:4.4-1c}}[\textwidth]{\begin{tikzpicture}

\def\circleA{(0:0) circle (7mm) node[index, above left=1mm and 2mm] (A) {$A$}}
\def\circleB{(0:5mm) circle (7mm) node[index, above right=1mm and 2mm] (B) {$B$}}
\def\circleC{(-60:5mm) circle (7mm) node[index, below=3mm] (C) {$C$}}

\begin{scope}
    \draw[fill=lightblue] \circleA;
    \draw \circleB;
    \draw \circleC +(0, -12mm) node[index] (label 1) {$A$};
\end{scope}

\node (center 1) at (barycentric cs:A=1,B=1,C=1) {};

\begin{scope}[xshift=28mm]
    \draw[fill=lightblue] \circleB \circleC +(0, -12mm) node[index] (label 2) {$(B\cup C)$};
    \draw \circleA;
\end{scope}

\node (center 2) at (barycentric cs:A=1,B=1,C=1) {};

\begin{scope}[xshift=56mm]
    \begin{scope}\clip \circleB;
        \draw[fill=lightblue] \circleA;
    \end{scope}
    \begin{scope}
        \clip \circleC;
        \draw[fill=lightblue] \circleA;
    \end{scope}
    \draw \circleA \circleB \circleC +(0, -12mm) node[index] (label 3) {$A\cap(B\cup C)$};
\end{scope}

\node (center 3) at (barycentric cs:A=1,B=1,C=1) {};

\begin{scope}[xshift=84mm]
    \begin{scope}
        \clip \circleB;
        \draw[fill=lightblue] \circleA;
    \end{scope}
    \draw \circleA \circleB \circleC +(0, -12mm) node[index] (label 4) {$(A\cap B)$};
\end{scope}

\node (center 4) at (barycentric cs:A=1,B=1,C=1) {};

\begin{scope}[xshift=112mm]
    \begin{scope}
        \clip \circleC;
        \draw[fill=lightblue] \circleA;
    \end{scope}
    \draw \circleA \circleB \circleC +(0, -12mm) node[index] (label 5) {$(A\cap C)$};
\end{scope}

\node (center 5) at (barycentric cs:A=1,B=1,C=1) {};

\path (center 1) -- (center 2) node[index, midway] (midlabel 1) {$\cap$}
      (center 2) -- (center 3) node[index, midway] (midlabel 2) {$=$}
      (center 3) -- (center 4) node[index, midway] (midlabel 3) {$=$}
      (center 4) -- (center 5) node[index, midway] (midlabel 4) {$\cup$};

\path (label 1) -| (midlabel 1) node[index, midway] {$\cap$}
      (label 2) -| (midlabel 2) node[index, midway] {$=$}
      (label 3) -| (midlabel 3) node[index, midway] {$=$}
      (label 4) -| (midlabel 4) node[index, midway] {$\cup$};

\end{tikzpicture}

\begin{tikzpicture}
    \node {17}
        child {node {13}
            child {node {8}
                child {node[discarded, label=left:{$i$}] {20} edge from parent[draw=none]}
                child {node[discarded] {25} edge from parent[draw=none]}
            }
            child {node {7}}
        }
        child {node {5}
            child {node {4}}
            child {node {2}
                child[missing]
                child {node[draw=none, fill=none, label=right:{}] {} edge from parent[draw=none]}
            }
        };
\end{tikzpicture}
}
    \par\vspace{10mm}
    \subcaptionbox{\label{fig:4.4-1d}}[\textwidth]{\begin{tikzpicture}

\def\circleA{(0:0) circle (7mm) node[index, above left=1mm and 2mm] (A) {$A$}}
\def\circleB{(0:5mm) circle (7mm) node[index, above right=1mm and 2mm] (B) {$B$}}
\def\circleC{(-60:5mm) circle (7mm) node[index, below=3mm] (C) {$C$}}

\begin{scope}
    \draw[fill=lightblue] \circleA;
    \draw \circleB;
    \draw \circleC +(0, -12mm) node[index] (label 1) {$A$};
\end{scope}

\node (center 1) at (barycentric cs:A=1,B=1,C=1) {};

\begin{scope}[xshift=28mm]
    \draw[fill=lightblue] \circleB \circleC +(0, -12mm) node[index] (label 2) {$(B\cup C)$};
    \draw \circleA;
\end{scope}

\node (center 2) at (barycentric cs:A=1,B=1,C=1) {};

\begin{scope}[xshift=56mm]
    \begin{scope}\clip \circleB;
        \draw[fill=lightblue] \circleA;
    \end{scope}
    \begin{scope}
        \clip \circleC;
        \draw[fill=lightblue] \circleA;
    \end{scope}
    \draw \circleA \circleB \circleC +(0, -12mm) node[index] (label 3) {$A\cap(B\cup C)$};
\end{scope}

\node (center 3) at (barycentric cs:A=1,B=1,C=1) {};

\begin{scope}[xshift=84mm]
    \begin{scope}
        \clip \circleB;
        \draw[fill=lightblue] \circleA;
    \end{scope}
    \draw \circleA \circleB \circleC +(0, -12mm) node[index] (label 4) {$(A\cap B)$};
\end{scope}

\node (center 4) at (barycentric cs:A=1,B=1,C=1) {};

\begin{scope}[xshift=112mm]
    \begin{scope}
        \clip \circleC;
        \draw[fill=lightblue] \circleA;
    \end{scope}
    \draw \circleA \circleB \circleC +(0, -12mm) node[index] (label 5) {$(A\cap C)$};
\end{scope}

\node (center 5) at (barycentric cs:A=1,B=1,C=1) {};

\path (center 1) -- (center 2) node[index, midway] (midlabel 1) {$\cap$}
      (center 2) -- (center 3) node[index, midway] (midlabel 2) {$=$}
      (center 3) -- (center 4) node[index, midway] (midlabel 3) {$=$}
      (center 4) -- (center 5) node[index, midway] (midlabel 4) {$\cup$};

\path (label 1) -| (midlabel 1) node[index, midway] {$\cap$}
      (label 2) -| (midlabel 2) node[index, midway] {$=$}
      (label 3) -| (midlabel 3) node[index, midway] {$=$}
      (label 4) -| (midlabel 4) node[index, midway] {$\cup$};

\end{tikzpicture}

\begin{tikzpicture}[
    level/.append style = {sibling distance=18\vertexsize/3^#1},
    total node/.append style={text width=8mm}
]

\node (root) {1}
    child {node (0) {1}
        child {node (00) {1}
            child {node (000) {}}
            child {node {}}
            child {node {}}
        }
        child {node (01) {1}
            child {node {}}
            child {node {}}
            child {node {}}
        }
        child {node (02) {1}
            child {node {}}
            child {node {}}
            child {node {}}
        }
    }
    child {node (1) {1}
        child {node (10) {1}
            child {node {}}
            child {node {}}
            child {node {}}
        }
        child {node (11) {1}
            child {node {}}
            child {node {}}
            child {node {}}
        }
        child {node (12) {1}
            child {node {}}
            child {node {}}
            child {node {}}
        }
    }
    child {node (2) {1}
        child {node (20) {1}
            child {node {}}
            child {node {}}
            child {node {}}
        }
        child {node (21) {1}
            child {node {}}
            child {node {}}
            child {node {}}
        }
        child {node (22) {1}
            child {node {}}
            child {node {}}
            child {node (222) {}}
        }
    };

\node[below=5mm of 000] (leaf 0) {$\Theta(1)$};
\foreach[count=\i] \x in {0, ..., 5} {
    \node[right=0pt of leaf \x] (leaf \i) {$\Theta(1)$};
}
\node[below=5mm of 222] (leaf m) {$\Theta(1)$};
\node[left=0pt of leaf m] (leaf m minus 1) {$\Theta(1)$};
\path (leaf 6) -- (leaf m minus 1) node[midway, font=\bfseries] {$\dots$};
\foreach \x in {0, ..., 6, m minus 1, m} {
    \draw[transition edge={draw=none}{very densely dashed}{0.5}] (leaf \x |- 000) -- (leaf \x.north);
}

\node[total node, right=of leaf m] (level h total) {$\Theta(3^n)$};
\node[total node] at (level h total |- 22) (level 2 total) {$3^2$};
\node[total node] at (level h total |- 2) (level 1 total) {$3$};
\node[total node] at (level h total |- root) (level 0 total) {$1$};

\path (level 2 total.east) -- (level h total.east) node[midway, xshift=-5mm, font=\bfseries] {$\vdots$};

\draw[level arrow, shorten <=5mm] (root) -- (level 0 total);
\draw[level arrow, shorten <=5mm] (2) -- (level 1 total);
\draw[level arrow, shorten <=5mm] (22) -- (level 2 total);
\draw[level arrow, shorten <=2mm] (leaf m) -- (level h total);

\draw[orange, decorate, decoration={brace, amplitude=8pt, mirror}]
    (leaf 0.south west) -- (leaf m.south east) node[midway, yshift=-6mm, text=black] (leaves total) {$3^n$};

\coordinate (total line end) at (leaves total -| level h total.east);
\draw (2 |- total line end) -- (total line end) node[at end, anchor=east, yshift=-4mm] {Total: $O(3^n)$};

\coordinate[left=4mm of leaf 0] (bottom);
\draw[<->, orange] (bottom) -- (bottom |- root) node[midway, fill=white, text=black] {$n$};

\end{tikzpicture}
}
    \caption{
        \textbf{(c)}\, A recursion tree for the recurrence $T(n)=4T(n/2)+n$.\,
        \textbf{(d)}\, A recursion tree for the recurrence $T(n)=3T(n-1)+1$.
    }
\end{figure}
