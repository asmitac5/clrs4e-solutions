\refFigure{4.4-1} shows the recursion trees for the recurrences solved in this exercise.
\vspace*{2\baselineskip}

\subexercise
For simplicity, assume that $n$ is an exact power of 2 and that the base case is $T(1)=\Theta(1)$.
At each level, the problem is reduced to a subproblem of exactly half the size, until the problem size hits $n=1$, so the height of the recursion tree is $\lg n$.
Each level of the tree has a single node which, at depth $i=0$, 1, \dots, $\lg n-1$, incurs the cost of $(n/2^i)^3=n^3\!/8^i$, and the leaf at depth $\lg n$ incurs the cost of $\Theta(1)$.
Therefore, the cost of the whole tree is
\begin{align*}
    T(n) &= \sum_{i=0}^{\lg n-1}n^3\!/8^i+\Theta(1) \\
    &< \sum_{i=0}^\infty n^3\!/8^i+\Theta(1) \\
    &= \frac{1}{1-(1/8)}\,n^3+\Theta(1) && \text{(by equation (A.7))} \\[1mm]
    &= (8/7)n^3+\Theta(1) \\
    &= O(n^3).
\end{align*}

To prove this result, we'll use the substitution method.
Let's adopt the inductive hypothesis that $T(n)\le cn^3$ for all $n\ge n_0$, where $c$, $n_0>0$ are constants, and assume that the inductive hypothesis holds for all values at least as big as $n_0$ and less than $n$.
Let $n\ge2n_0$, so that $T(n/2)\le c(n/2)^3$.
Substituting into the recurrence yields
\begin{align*}
    T(n) &\le c(n/2)^3+n^3 \\
    &= cn^3\!/8+n^3 \\
    &= (c/8+1)n^3 \\
    &\le cn^3,
\end{align*}
where the last step holds as long as $c/8+1\le c$, or $c\ge8/7$.

Let's pick $n_0=1$.
Picking $c=\max{8/7,T(1)}$ satisfies $T(1)\le c\cdot1^3$, establishing the inductive hypothesis for the base case.

We've shown that $T(n)\le cn^3$ for all $n\ge1$, which implies that the solution to the recurrence is $T(n)=O(n^3)$.

\subexercise
Assume that $n$ is an exact power of 3 and that $T(1)=\Theta(1)$.
The subproblem size for a node at depth $i$ is $n/3^i$, and the bottom level of the tree is at depth $i$ such that $n/3^i=1$ or, equivalently, $i=\log_3n$.
Each level below the top has four times as many nodes as the level above, and so the number of nodes at depth $i$ is $4^i$.
The total cost of all nodes at a given depth $i$ is $4^i(n/3^i)=(4/3)^in$.
The bottom level contains $4^{\log_3n}=n^{\log_34}$ leaves, each contributing $\Theta(1)$, leading to a total leaf cost of $\Theta(n^{\log_34})$.
Therefore, the cost of the entire tree is
\begin{align*}
    T(n) &= \sum_{i=0}^{\log_3n-1}(4/3)^in+\Theta(n^{\log_34}) \\[1mm]
    &= \frac{(4/3)^{\log_3n}-1}{4/3-1}\,n+\Theta(n^{\log_34}) && \text{(by equation (A.6))} \\[1mm]
    &< 3n(4/3)^{\log_3n}+\Theta(n^{\log_34}) \\[1mm]
    &= 3n\,\frac{n^{\log_34}}{n^{\log_33}}+\Theta(n^{\log_34}) && \text{(by equation (3.21))} \\[2mm]
    &= 3n^{\log_34}+\Theta(n^{\log_34}) \\
    &= O(n^{\log_34}).
\end{align*}

To verify this bound using the substitution method, we adopt the inductive hypothesis that $T(n)\le cn^{\log_34}-dn$ for all $n\ge n_0$, where $c$, $n_0>0$ and $d\ge0$ are some constants.
Assume that the inductive hypothesis holds for all numbers at least as big as $n_0$ and less than $n$.
If we let $n\ge3n_0$, we have $T(n/3)\le c(n/3)^{\log_34}-d(n/3)$.
Substituting into the recurrence yields
\begin{align*}
    T(n) &\le 4\left(c(n/3)^{\log_34}-d(n/3)\right)+n \\
    &= 4cn^{\log_34}\!/3^{\log_34}-4dn/3+n \\
    &= 4cn^{\log_34}\!/4^{\log_33}-4dn/3+n \\
    &= cn^{\log_34}-dn+n(1-d/3) \\
    &\le cn^{\log_34}-dn,
\end{align*}
where the last step holds as long as $1-d/3\le0$, or $d\ge3$.

Let's pick $n_0=1$.
Picking $c=\max{T(1)+d,(T(2)+2d)/2^{\log_34}}$ satisfies both conditions $T(1)\le c\cdot1^{\log_34}-d\cdot1$ and $T(2)\le c\cdot2^{\log_34}-d\cdot2$, establishing the inductive hypothesis for the base cases.

We've shown that $T(n)\le cn^{\log_34}-dn\le cn^{\log_34}$ for all $n\ge1$, which implies that the solution to the recurrence is $T(n)=O(n^{\log_34})$.

\subexercise
Assume that $n$ is an exact power of 2 and that $T(1)=\Theta(1)$.
The height of the recursion tree is $\lg n$.
At depth $i$, for $i=0$, 1, \dots, $\lg n-1$, there are $4^i$ nodes, each contributing $n/2^i$, so the total cost of all nodes at depth $i$ is $4^i(n/2^i)=2^in$.
The bottom level, at depth $\lg n$, contains $4^{\lg n}=n^2$ leaves, so the total leaf cost is $\Theta(n^2)$.
Thus, we get
\begin{align*}
    T(n) &= \sum_{i=0}^{\lg n-1}2^in+\Theta(n^2) \\
    &= \frac{2^{\lg n}-1}{2-1}\,n+\Theta(n^2) && \text{(by equation (A.6))} \\[1mm]
    &< 2^{\lg n}n+\Theta(n^2) \\
    &= n^2+\Theta(n^2) \\
    &= O(n^2).
\end{align*}

To formally prove this bound we adopt the inductive hypothesis that $T(n)\le cn^2-dn$ for all $n\ge n_0$, where $c$, $n_0>0$ and $d\ge0$ are constants, and assume that the inductive hypothesis holds for all numbers at least as big as $n_0$ and less than $n$.
When $n\ge2n_0$ it holds that $T(n/2)\le c(n/2)^2-d(n/2)$, so
\begin{align*}
    T(n) &\le 4(c(n/2)^2-d(n/2))+n \\
    &= cn^2-2dn+n \\
    &= cn^2-dn+n(1-d) \\
    &\le cn^2-dn && \text{(as long as $d\ge1$)}.
\end{align*}

If we now let $n_0=1$, we can choose $c=T(1)+d$ to satisfy $T(1)\le c\cdot1^2-d\cdot1$, establishing the bound for the base case.

We've shown that $T(n)\le cn^2-dn\le cn^2$ for all $n\ge1$, which implies that the solution to the recurrence is $T(n)=O(n^2)$.

\subexercise
Assume that $T(1)=\Theta(1)$.
The height of the recursion tree is $n$.
At depth $i$, for $i=0$, 1, \dots, $n-1$, there are $3^i$ nodes incurring a unit cost each, so the total cost of all nodes at depth $i$ is $3^i$.
The bottom level, at depth $n$, contains $3^n$ leaves, so the total leaf cost is $\Theta(3^n)$.
Thus, we get
\begin{align*}
    T(n) &= \sum_{i=0}^{n-1}3^i+\Theta(3^n) \\[1mm]
    &= \frac{3^n-1}{3-1}+\Theta(3^n) && \text{(by equation (A.6))} \\[1mm]
    &< 2\cdot3^n+\Theta(3^n) \\
    &= O(3^n).
\end{align*}

To prove this bound we adopt the inductive hypothesis that $T(n)\le c3^n-d$ for all $n\ge n_0$, where $c$, $n_0>0$ and $d\ge0$ are constants, and assume that the inductive hypothesis holds for all numbers at least as big as $n_0$ and less than $n$.
When $n\ge n_0+1$ it holds that $T(n-1)\le c3^{n-1}-d$, so
\begin{align*}
    T(n) &\le 3(c3^{n-1}-d)+1 \\
    &= c3^n-3d+1 \\
    &= c3^n-d+(1-2d) \\
    &\le c3^n-d && \text{(as long as $d\ge1/2$)}.
\end{align*}

If we now let $n_0=1$, we can choose $c=(T(1)+d)/3$ to satisfy $T(1)\le c\cdot3^1-d$, establishing the bound for the base case.

We've shown that $T(n)\le c3^n-d\le c3^n$ for all $n\ge1$, which implies that the solution to the recurrence is $T(n)=O(3^n)$.

\begin{figure}[!htb]
    \subcaptionbox{\label{fig:4.4-1a}}[\textwidth]{\subimport{figures/1/}{a.tikz}}
    \par\vspace{10mm}
    \subcaptionbox{\label{fig:4.4-1b}}[\textwidth]{\subimport{figures/1/}{b.tikz}}
    \caption{
        \textbf{(a)}\, A recursion tree for the recurrence $T(n)=T(n/2)+n^3$.\,
        \textbf{(b)}\, A recursion tree for the recurrence $T(n)=4T(n/3)+n$.
    } \label{fig:4.4-1}
\end{figure}

\begin{figure}[htb!]\ContinuedFloat
    \subcaptionbox{\label{fig:4.4-1c}}[\textwidth]{\subimport{figures/1/}{c.tikz}}
    \par\vspace{10mm}
    \subcaptionbox{\label{fig:4.4-1d}}[\textwidth]{\subimport{figures/1/}{d.tikz}}
    \caption{
        \textbf{(c)}\, A recursion tree for the recurrence $T(n)=4T(n/2)+n$.\,
        \textbf{(d)}\, A recursion tree for the recurrence $T(n)=3T(n-1)+1$.
    }
\end{figure}
