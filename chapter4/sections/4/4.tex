We'll use the recursion-tree method to guess an upper bound on $T(n)$.
Let's assume that $0<\alpha\le1/2$, because the case when $1/2\le\alpha<1$ is symmetric, and as we'll see, the choice of $\alpha$ doesn't affect the order of growth of an upper bound on $T(n)$.
\refFigure{4.4-4} shows such a tree.
\begin{figure}[htb]
    \begin{tikzpicture}
    \pgfkeys{/pgf/number format/.cd,fixed}
    \foreach \x in {-4, ..., 4} {
        \foreach \y in {-4, ..., 4} {
            \pgfmathsetmacro\absx{abs(\x)}
            \pgfmathsetmacro\absy{abs(\y)}
            \pgfmathsetmacro\d{max(\absx,\absy)}
            \pgfmathsetmacro\D{(2 * \d - 1)^2 - 1}
            \pgfmathsetmacro\g{(\d == \x && \x != \absy ? \D + \d + \y :
                               (\d == \y && \y != 0 ? \D + 3 * \d - \x :
                               (\d == -\x && \x != \absy ? \D + 5 * \d - \y :
                               (\d == -\y && \y != 0 ? \D + 7 * \d + \x : 0))))}
            \pgfmathsetmacro\nodevalue{int(\g)}
            \node[inner sep=1pt] at (7*\x mm, 7*\y mm) (node \nodevalue) {\nodevalue};
        }
    }
    
    \node[right=3mm of node 80, inner sep=1pt] (node 81) {$\dots$};
    \foreach [count=\i] \j in {0, ..., 80} {
        \draw (node \i) -- (node \j);
    }
    
    \draw[->] (node 52.east) + (1mm,0mm) -- +(10mm, 0mm) node [above left=1mm] {$x$};
    \draw[->] (node 60.north) + (0mm, 1mm) -- +(0mm, 10mm) node [below left=1mm] {$y$};
    \draw[-]  (node 68.west) + (-1mm, 0mm) -- +(-7mm, 0mm);
    \draw[-]  (node 76.south) + (0mm, -1mm) -- +(0mm, -7mm);
\end{tikzpicture}

    \caption{A recursion tree for the recurrence $T(n)=T(\alpha n)+T((1-\alpha)n)+cn$, where $0<\alpha\le1/2$.} \label{fig:4.4-4}
\end{figure}

Let $n_0>0$ be the implicit threshold constant such that $T(n)=\Theta(1)$ for $0<n<n_0$, and let $c$ represent the upper-bound constant hidden by the $\Theta(n)$ term for $n\ge n_0$.
The root-to-leaf path formed by right-going edges is the longest, where the node at depth $i$ represents a subproblem of size $(1-\alpha)^in$.
Let $h$ be the height of the tree.
At depth $h$ the rightmost node is a leaf, so $(1-\alpha)^hn<n_0\le(1-\alpha)^{h-1}n$, which gives $h=\lfloor\log_{1/(1-\alpha)}(n/n_0)\rfloor+1$, and thus $h=\Theta(\lg n)$.
The recursion divides the problem into two subproblems, whose sizes sum up to the size of the problem.
Since the cost incurred by a problem of size $m$ is $cm$, the sum of costs incurred by the subproblems is also $cm$.
Therefore, the total cost of all nodes at any depth is bound by $cn$.
Summing the costs of internal nodes across each level, we have at most $cn$ per level times $h=\Theta(\lg n)$ for a total cost of $O(n\lg n)$ for all internal nodes.

The number of leaves of the tree is described by the following recurrence:
\[
    L(n) =
    \ccases{
        1 & \text{if $n<n_0$}, \\
        L(\alpha n)+L((1-\alpha)n) & \text{if $n\ge n_0$}.
    }
\]
We can apply the substitution method to show it has solution $L(n)=O(n)$ for any $0<\alpha<1$.
Using the inductive hypothesis $L(n)\le dn$ for some constant $d>0$, and assuming that the inductive hypothesis holds for all values less than $n$, we have
\begin{align*}
    L(n) &= L(\alpha n)+L((1-\alpha)n) \\
    &\le d\alpha n+d(1-\alpha)n \\
    &= dn,
\end{align*}
which holds for any $d>0$ and any $0<\alpha<1$.
Picking $d=1$ suffices to handle the base case $L(1)=1$ for $0<n<n_0$.

Returning to recurrence $T(n)$, we obtain its upper bound of $O(n\lg n)+O(n)=O(n\lg n)$.
