Choose any $\phi\ge1$.
Let $1\le\psi\le\phi$, and thus $1^2\le\psi^2\le\phi^2$.
Multiplying all sides by $n^2$, where $n>0$, yields
\[
    n^2 \le \psi^2n^2 \le \phi^2n^2
\]
for all $n>0$.
Since $n^2/(\phi^2+1)<n^2$ and $\phi^2n^2<(\phi^2+1)n^2$, we can let $d=\phi^2+1>1$ to get
\[
    f(n)/d < f(\psi n) < df(n).
\]
Thus, the function $f(n)=n^2$ satisfies the polynomial-growth condition.

Now for $f(n)=2^n$ let $\phi=\psi=2$.
We have
\begin{align*}
    f(\psi n) &= 2^{\psi n} \\
    &= 2^{2n} \\
    &= 2^n\cdot2^n \\
    &\le d2^n \\
    &= df(n),
\end{align*}
as long as $d\ge2^n$, so there is no suitable constant $d$ for the second to last step to hold for all sufficiently large $n$.
We conclude that $f(n)=2^n$ doesn't satisfy the polynomial-growth condition.
