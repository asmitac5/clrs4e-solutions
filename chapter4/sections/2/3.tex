For convenience, we'll assume that $n$ is an exact power of 3.
Similarly to equation (4.2), let's partition the matrices into nine $n/3\times n/3$ submatrices.
We can then treat the input matrices as two $3\times3$ matrices that we multiply using $k$ multiplications.
Each of those is in fact a multiplication of two $n/3\times n/3$ matrices and can be performed by running the algorithm recursively.

Let $T(n)$ be the worst-case time to multiply two $n\times n$ matrices by this algorithm.
If we did the partitioning using index calculations, then we get the recurrence
\[
    T(n) = kT(n/3)+\Theta(1).
\]
To solve it, we'll use the master method.
We have $a=k$, $b=3$, and driving function $f(n)=\Theta(1)$.
Since $n^{\log_ba}=n^{\log_3k}$ and $f(n)=O(n^{\log_3k-\epsilon})$ for $0<\epsilon\le\log_3k$, we apply case~1 of the master theorem to get $T(n)=\Theta(n^{\log_3k})$.
According to \refProblem{3-1}(d), $T(n)=o(n^{\lg7})$ if $\log_3k<\lg7$, which gives us $k<3^{\lg7}\approx21.85$.
Hence, the largest integer value for $k$ is 21 and then the running time of the algorithm is $T(n)=\Theta(n^{\log_321})=O(n^{2.78})$.
