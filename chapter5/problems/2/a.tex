\begin{codebox}
\Procname{$\proc{Random-Search}(A,n,x)$}
\li let $\id{picked}[1\subarr n]$ be a new array
\li \For $i\gets1$ \To $n$
\li     \Do $\id{picked}[i]\gets\const{false}$
        \End
\li $k\gets0$
\li \While $k<n$ \label{li:random-search-while-begin}
\li     \Do $i\gets\proc{Random}(1,n)$
\li         \If $A[i]\isequal x$
\li             \Then \Return $i$
                \End
\li         \If not $\id{picked}[i]$
\li             \Then $\id{picked}[i]\gets\const{true}$
\li                 $k\gets k+1$
                \End
        \End \label{li:random-search-while-end}
\li \Return \nil
\end{codebox}

The algorithm uses the auxiliary array $\id{picked}[1\subarr n]$ of boolean values for keeping track of the information whether it has already picked a given index, and the variable $k$ for the number of picked indices so far.
In each iteration of the \kw{while} loop in lines \ref{li:random-search-while-begin}--\ref{li:random-search-while-end} the algorithm picks a random index into $A$ and compares $A[i]$ with $x$.
If $x$ is found at position $i$, the algorithm immediately returns that position.
Otherwise, if $A[i]$ hasn't been examined in previous iterations, the algorithm increments $k$.
If array $A$ doesn't contain the element $x$, then after checking all $n$ indices at least once, the algorithm terminates by returning the special value \nil.
