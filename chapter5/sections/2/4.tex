Let $X_1$ and $X_2$ be the random variables representing the values on each die.
Both variables have the same probability distribution, so their expectations are identical.
We have
\begin{align*}
    \E{X_1} &= \sum_{x=1}^6x\Pr{X_1=x} \\
    &= (1/6)\cdot(1+2+3+4+5+6) \\
    &= 21/6 \\
    &= 3.5.
\end{align*}
Therefore, the expected value of the sum of two 6-sided dice is
\begin{align*}
    \E{X_1+X_2} &= \E{X_1}+\E{X_2} && \text{(by equation (C.24), linearity of expectation)} \\
    &= 3.5+3.5 \\
    &= 7.
\end{align*}

For the second scenario, consider the random variables $Y_1$ and $Y_2$ representing the values of the first die, and the second die, respectively.
Note that $Y_1$ has the same probability distribution as $X_1$ (and $X_2$) from the first scenario, so $\E{Y_1}=\E{X_1}=3.5$.
On the other hand, $Y_2$ is dependent on $Y_1$, taking the identical values as $Y_1$, so $Y_2=Y_1$.
Therefore,
\begin{align*}
    \E{Y_1+Y_2} &= \E{2Y_1} \\
    &= 2\E{Y_1} && \text{(by equation (C.25))} \\
    &= 2\cdot3.5 \\
    &= 7.
\end{align*}

Let $Z_1$ and $Z_2$ be the random variables indicating the values of both dice in the third scenario.
Note that $Z_2=7-Z_1$, so
\begin{align*}
    \E{Z_1+Z_2} &= \E{Z_1+7-Z_1} \\
    &= \E{7} \\
    &= 7.
\end{align*}
