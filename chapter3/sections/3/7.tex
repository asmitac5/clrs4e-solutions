\clarification[4]{$\phi$ and $\widehat\phi$ should be given by equations (3.32) and (3.33), respectively.}
We have
\begin{align*}
    \phi^2 &= \left(\frac{1+\sqrt{5}}{2}\right)^2 \\
    &= \frac{1+2\sqrt{5}+5}{4} \\[1mm]
    &= \frac{3+\sqrt{5}}{2} \\[1mm]
    &= \frac{1+\sqrt{5}}{2}+1 \\
    &= \phi+1
\end{align*}
and
\begin{align*}
    \widehat\phi^2 &= \left(\frac{1-\sqrt{5}}{2}\right)^2 \\
    &= \frac{1-2\sqrt{5}+5}{4} \\[1mm]
    &= \frac{3-\sqrt{5}}{2} \\[1mm]
    &= \frac{1-\sqrt{5}}{2}+1 \\
    &= \widehat\phi+1,
\end{align*}
so both $\phi$ and $\widehat\phi$ are the roots of the equation $x^2=x+1$, and therefore are the golden ratio and its conjugate, respectively.
