The fact trivially holds for binary trees with at most three nodes, so suppose that $n\ge4$.
We define a \concept{branch} of a binary tree as a path $\langle x_0$, $x_1$, \dots, $x_k\rangle$, where $x_0$ is the root, $x_k$ is a leaf, and for $i=0$, 1, \dots, $k-1$, $x_{i+1}$ is the root of the subtree of $x_i$ with more nodes.
Note that a tree may have more than one branch.
Let's denote by $s(x)$ the number of nodes in the subtree rooted at $x$.
For every $i=0$, 1, \dots, $k-1$,
\begin{equation} \label{eq:subtree-sizes-on-branch}
    s(x_i) \le 2s(x_{i+1})+1.
\end{equation}
Of course, the sequence $\langle s(x_0)$, $s(x_1)$, \dots, $s(x_k)\rangle$ decreases from $n$ to 1, so there exists an integer $j$, where $0<j<k$, such that $s(x_j)>n/4$ and $s(x_{j+1})\le n/4$.
Thus, by removing edge $(x_{j-1},x_j)$, we partition the nodes of the tree into two sets of sizes
\begin{align*}
    s(x_j) &\le 2s(x_{j+1})+1 && \text{(by inequality \eqref{eq:subtree-sizes-on-branch})} \\
    &\le 2n/4+1 \\
    &\le 3n/4 && \text{(since $n\ge4$)}
\end{align*}
and
\begin{align*}
    n-s(x_j) &< n-n/4 \\
    &= 3n/4.
\end{align*}
