For the purpose of time calculations, we define a month as 30 days and a year as 365 days.
Note that multiplying the 30-day month expression by 12 to derive a year would be inaccurate.
The size $n$ of a problem solvable within the given time $t$ (in microseconds) satisfies the inequality $f(n)\le t$, and we can solve it algebraically or using a numerical method.

The results are collected in the table in \refFigure{1-1}.
Values above $10^6$ are approximate.

\begin{figure}[htb]
    \[
        \renewcommand{\arraystretch}{1.3}
        \begin{array}{c|c|c|c|c|c|c|c|}
            & \text{1 second} & \text{1 minute} & \text{1 hour} & \text{1 day} & \text{1 month} & \text{1 year} & \text{1 century} \\
            \hline
            \lg n & 10^{301{,}030} & 10^{1.81\cdot10^7} & 10^{1.08\cdot10^9} & 10^{2.6\cdot10^{10}} & 10^{7.8\cdot10^{11}} & 10^{9.49\cdot10^{12}} & 10^{9.49\cdot10^{14}} \\
            \hline
            \sqrt{n} & 10^{12} & 3.6\cdot10^{15} & 1.3\cdot10^{19} & 7.46\cdot10^{21} & 6.72\cdot10^{24} & 9.95\cdot10^{26} & 9.95\cdot10^{30} \\
            \hline
            n & 10^6 & 6\cdot10^7 & 3.6\cdot10^9 & 8.64\cdot10^{10} & 2.59\cdot10^{12} & 3.15\cdot10^{13} & 3.15\cdot10^{15} \\
            \hline
            n\lg n & 62{,}746 & 2.8\cdot10^6 & 1.33\cdot10^8 & 2.76\cdot10^9 & 7.19\cdot10^{10} & 7.98\cdot10^{11} & 6.86\cdot10^{13} \\
            \hline
            n^2 & 1{,}000 & 7{,}745 & 60{,}000 & 293{,}938 & 1.61\cdot10^6 & 5.62\cdot10^6 & 5.62\cdot10^7 \\
            \hline
            n^3 & 100 & 391 & 1{,}532 & 4{,}420 & 13{,}736 & 31{,}593 & 146{,}645 \\
            \hline
            2^n & 19 & 25 & 31 & 36 & 41 & 44 & 51 \\
            \hline
            n! & 9 & 11 & 12 & 13 & 15 & 16 & 17 \\
            \hline
        \end{array}
    \]
    \caption{Largest sizes of problems solvable in a given time by an algorithm of a given time efficiency.} \label{fig:1-1}
\end{figure}
