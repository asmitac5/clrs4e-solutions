See \refFigure{6.5-1}.
\begin{figure}[htb]
    \captionsetup[subfigure]{}
    \subcaptionbox{\label{fig:6.5-1a}}[0.5\textwidth]{\subimport{figures/1/}{a.tikz}}
    \subcaptionbox{\label{fig:6.5-1b}}[0.5\textwidth]{\subimport{figures/1/}{b.tikz}}
    \caption{The operation of \proc{Max-Heap-Extract-Max} on the heap $A=\langle$15, 13, 9, 5, 12, 8, 7, 4, 0, 6, 2, 1$\rangle$.\,
    \textbf{(a)}\, The max-heap at the start of the procedure.
    In line~1 the variable \id{max} is set to the root of the heap holding the maximum key 15, which will be returned at the end.
    The root is updated with the node of key 1 and the size of the heap is reduced by 1.\,
    \textbf{(b)}\, The max-heap after the max-heap property is restored in line~4.} \label{fig:6.5-1}
\end{figure}
