See \refFigure{6.3-1}.
\begin{figure}[htb]
    \captionsetup[subfigure]{}
    \subcaptionbox{\label{fig:6.3-1a}}[0.5\textwidth]{\begin{tikzpicture}

\def\circleA{(0:0) circle (7mm) node[index, above left=1mm and 2mm] (A) {$A$}}
\def\circleB{(0:5mm) circle (7mm) node[index, above right=1mm and 2mm] (B) {$B$}}
\def\circleC{(-60:5mm) circle (7mm) node[index, below=3mm] (C) {$C$}}

\begin{scope}
    \draw[fill=lightblue] \circleA;
    \draw \circleB;
    \draw \circleC +(0, -12mm) node[index] (label 1) {$A$};
\end{scope}

\node (center 1) at (barycentric cs:A=1,B=1,C=1) {};

\begin{scope}[xshift=28mm]
    \draw[fill=lightblue] \circleB \circleC +(0, -12mm) node[index] (label 2) {$(B\cup C)$};
    \draw \circleA;
\end{scope}

\node (center 2) at (barycentric cs:A=1,B=1,C=1) {};

\begin{scope}[xshift=56mm]
    \begin{scope}\clip \circleB;
        \draw[fill=lightblue] \circleA;
    \end{scope}
    \begin{scope}
        \clip \circleC;
        \draw[fill=lightblue] \circleA;
    \end{scope}
    \draw \circleA \circleB \circleC +(0, -12mm) node[index] (label 3) {$A\cap(B\cup C)$};
\end{scope}

\node (center 3) at (barycentric cs:A=1,B=1,C=1) {};

\begin{scope}[xshift=84mm]
    \begin{scope}
        \clip \circleB;
        \draw[fill=lightblue] \circleA;
    \end{scope}
    \draw \circleA \circleB \circleC +(0, -12mm) node[index] (label 4) {$(A\cap B)$};
\end{scope}

\node (center 4) at (barycentric cs:A=1,B=1,C=1) {};

\begin{scope}[xshift=112mm]
    \begin{scope}
        \clip \circleC;
        \draw[fill=lightblue] \circleA;
    \end{scope}
    \draw \circleA \circleB \circleC +(0, -12mm) node[index] (label 5) {$(A\cap C)$};
\end{scope}

\node (center 5) at (barycentric cs:A=1,B=1,C=1) {};

\path (center 1) -- (center 2) node[index, midway] (midlabel 1) {$\cap$}
      (center 2) -- (center 3) node[index, midway] (midlabel 2) {$=$}
      (center 3) -- (center 4) node[index, midway] (midlabel 3) {$=$}
      (center 4) -- (center 5) node[index, midway] (midlabel 4) {$\cup$};

\path (label 1) -| (midlabel 1) node[index, midway] {$\cap$}
      (label 2) -| (midlabel 2) node[index, midway] {$=$}
      (label 3) -| (midlabel 3) node[index, midway] {$=$}
      (label 4) -| (midlabel 4) node[index, midway] {$\cup$};

\end{tikzpicture}

\begin{tikzpicture}[total node/.append style={text width=12mm}]

\node (root) {$n^3$}
	child {node (0) {$\bigl(\frac{n}{2}\bigr)^3$}
		child {node (00) {$\bigl(\frac{n}{4}\bigr)^3$}
			child {node (000) {}}
		}
	};

\node[below=5mm of 000] (leaf) {$\Theta(1)$};
\draw[transition edge={draw=none}{very densely dashed}{0.5}] (000) -- (leaf);

\node[total node, right=of leaf] (level h total) {$\Theta(1)$};
\node[total node] at (level h total |- 00) (level 2 total) {$\bigl(\frac{1}{8}\bigr)^2n^3$};
\node[total node] at (level h total |- 0) (level 1 total) {$\frac{1}{8}\,n^3$};
\node[total node] at (level h total |- root) (level 0 total) {$n^3$};

\path (level 2 total.east) -- (level h total.east) node[midway, xshift=-5mm, font=\bfseries] {$\vdots$};

\draw[level arrow, shorten <=2mm] (root) -- (level 0 total);
\draw[level arrow, shorten <=2mm] (0) -- (level 1 total);
\draw[level arrow, shorten <=2mm] (00) -- (level 2 total);
\draw[level arrow, shorten <=2mm] (leaf) -- (level h total);

\coordinate[below=5mm of leaf] (leaves total);
\coordinate (total line end) at (leaves total -| level h total.east);
\draw (0 |- total line end) -- (total line end) node[at end, anchor=east, yshift=-4mm] {Total: $O(n^3)$};

\coordinate[left=4mm of leaf] (bottom);
\draw[<->, orange] (bottom) -- (bottom |- root) node[midway, fill=white, text=black] {$\lg n$};

\end{tikzpicture}
}
    \subcaptionbox{\label{fig:6.3-1b}}[0.5\textwidth]{\begin{tikzpicture}

\def\circleA{(0:0) circle (7mm) node[index, above left=1mm and 2mm] (A) {$A$}}
\def\circleB{(0:5mm) circle (7mm) node[index, above right=1mm and 2mm] (B) {$B$}}
\def\circleC{(-60:5mm) circle (7mm) node[index, below=3mm] (C) {$C$}}

\begin{scope}
    \draw[fill=lightblue] \circleA;
    \draw \circleB;
    \draw \circleC +(0, -12mm) node[index] (label 1) {$A$};
\end{scope}

\node (center 1) at (barycentric cs:A=1,B=1,C=1) {};

\begin{scope}[xshift=28mm]
    \draw[fill=lightblue] \circleB \circleC +(0, -12mm) node[index] (label 2) {$(B\cup C)$};
    \draw \circleA;
\end{scope}

\node (center 2) at (barycentric cs:A=1,B=1,C=1) {};

\begin{scope}[xshift=56mm]
    \begin{scope}\clip \circleB;
        \draw[fill=lightblue] \circleA;
    \end{scope}
    \begin{scope}
        \clip \circleC;
        \draw[fill=lightblue] \circleA;
    \end{scope}
    \draw \circleA \circleB \circleC +(0, -12mm) node[index] (label 3) {$A\cap(B\cup C)$};
\end{scope}

\node (center 3) at (barycentric cs:A=1,B=1,C=1) {};

\begin{scope}[xshift=84mm]
    \begin{scope}
        \clip \circleB;
        \draw[fill=lightblue] \circleA;
    \end{scope}
    \draw \circleA \circleB \circleC +(0, -12mm) node[index] (label 4) {$(A\cap B)$};
\end{scope}

\node (center 4) at (barycentric cs:A=1,B=1,C=1) {};

\begin{scope}[xshift=112mm]
    \begin{scope}
        \clip \circleC;
        \draw[fill=lightblue] \circleA;
    \end{scope}
    \draw \circleA \circleB \circleC +(0, -12mm) node[index] (label 5) {$(A\cap C)$};
\end{scope}

\node (center 5) at (barycentric cs:A=1,B=1,C=1) {};

\path (center 1) -- (center 2) node[index, midway] (midlabel 1) {$\cap$}
      (center 2) -- (center 3) node[index, midway] (midlabel 2) {$=$}
      (center 3) -- (center 4) node[index, midway] (midlabel 3) {$=$}
      (center 4) -- (center 5) node[index, midway] (midlabel 4) {$\cup$};

\path (label 1) -| (midlabel 1) node[index, midway] {$\cap$}
      (label 2) -| (midlabel 2) node[index, midway] {$=$}
      (label 3) -| (midlabel 3) node[index, midway] {$=$}
      (label 4) -| (midlabel 4) node[index, midway] {$\cup$};

\end{tikzpicture}

\begin{tikzpicture}[
	level/.append style = {sibling distance=18\vertexsize/4^#1},
	total node/.append style={text width=15mm}
]

\node (root) {$n$}
	child {node (0) {$\frac{n}{3}$}
		child {node (00) {$\frac{n}{9}$}
			child {node (000) {}}
			child {node {}}
			child {node {}}
			child {node {}}
		}
		child {node (01) {$\frac{n}{9}$}
			child {node {}}
			child {node {}}
			child {node {}}
			child {node {}}	
		}
		child {node (02) {$\frac{n}{9}$}
			child {node {}}
			child {node {}}
			child {node {}}
			child {node {}}
		}
		child {node (03) {$\frac{n}{9}$}
			child {node {}}
			child {node {}}
			child {node {}}
			child {node {}}
		}
	}
	child {node (1) {$\frac{n}{3}$}
		child {node (10) {$\frac{n}{9}$}
			child {node {}}
			child {node {}}
			child {node {}}
			child {node {}}
		}
		child {node (11) {$\frac{n}{9}$}
			child {node {}}
			child {node {}}
			child {node {}}
			child {node {}}
		}
		child {node (12) {$\frac{n}{9}$}
			child {node {}}
			child {node {}}
			child {node {}}
			child {node {}}
		}
		child {node (13) {$\frac{n}{9}$}
			child {node {}}
			child {node {}}
			child {node {}}
			child {node {}}
		}
	}
	child {node (2) {$\frac{n}{3}$}
		child {node (20) {$\frac{n}{9}$}
			child {node {}}
			child {node {}}
			child {node {}}
			child {node {}}
		}
		child {node (21) {$\frac{n}{9}$}
			child {node {}}
			child {node {}}
			child {node {}}
			child {node {}}
		}
		child {node (22) {$\frac{n}{9}$}
			child {node {}}
			child {node {}}
			child {node {}}
			child {node {}}
		}
		child {node (23) {$\frac{n}{9}$}
			child {node {}}
			child {node {}}
			child {node {}}
			child {node {}}
		}
	}
	child {node (3) {$\frac{n}{3}$}
		child {node (30) {$\frac{n}{9}$}
			child {node {}}
			child {node {}}
			child {node {}}
			child {node {}}
		}
		child {node (31) {$\frac{n}{9}$}
			child {node {}}
			child {node {}}
			child {node {}}
			child {node {}}
		}
		child {node (32) {$\frac{n}{9}$}
			child {node {}}
			child {node {}}
			child {node {}}
			child {node {}}
		}
		child {node (33) {$\frac{n}{9}$}
			child {node {}}
			child {node {}}
			child {node {}}
			child {node (333) {}}
		}
	};
	
\node[below=5mm of 000] (leaf 0) {$\Theta(1)$};
\foreach[count=\i] \x in {0, ..., 5} {
	\node[right=0pt of leaf \x] (leaf \i) {$\Theta(1)$};
}
\node[below=5mm of 333] (leaf m) {$\Theta(1)$};
\node[left=0pt of leaf m] (leaf m minus 1) {$\Theta(1)$};
\node[left=0pt of leaf m minus 1] (leaf m minus 2) {$\Theta(1)$};
\path (leaf 6) -- (leaf m minus 2) node[midway, font=\bfseries] {$\dots$};
\foreach \x in {0, ..., 6, m minus 2, m minus 1, m} {
	\draw[transition edge={draw=none}{very densely dashed}{0.5}] (leaf \x |- 000) -- (leaf \x.north);
}

\node[total node, right=of leaf m] (level h total) {$\Theta(n^{\log_34})$};
\node[total node] at (level h total |- 22) (level 2 total) {$\bigl(\frac{4}{3}\bigr)^2n$};
\node[total node] at (level h total |- 2) (level 1 total) {$\frac{4}{3}\,n$};
\node[total node] at (level h total |- root) (level 0 total) {$n$};

\path (level 2 total.east) -- (level h total.east) node[midway, xshift=-5mm, font=\bfseries] {$\vdots$};

\draw[level arrow, shorten <=5mm] (root) -- (level 0 total);
\draw[level arrow, shorten <=5mm] (3) -- (level 1 total);
\draw[level arrow, shorten <=5mm] (33) -- (level 2 total);
\draw[level arrow, shorten <=2mm] (leaf m) -- (level h total);

\draw[orange, decorate, decoration={brace, amplitude=8pt, mirror}]
	(leaf 0.south west) -- (leaf m.south east) node[midway, yshift=-5mm, text=black] (leaves total) {$4^{\log_3n}=n^{\log_34}$};

\coordinate (total line end) at (leaves total -| level h total.east);
\draw (3 |- total line end) -- (total line end) node[at end, anchor=east, yshift=-4mm] {Total: $O(n^{\log_34})$};

\coordinate[left=4mm of leaf 0] (bottom);
\draw[<->, orange] (bottom) -- (bottom |- root) node[midway, fill=white, text=black] {$\log_3n$};

\end{tikzpicture}
}
    \par\vspace{7mm}
    \subcaptionbox{\label{fig:6.3-1c}}[0.5\textwidth]{\begin{tikzpicture}

\def\circleA{(0:0) circle (7mm) node[index, above left=1mm and 2mm] (A) {$A$}}
\def\circleB{(0:5mm) circle (7mm) node[index, above right=1mm and 2mm] (B) {$B$}}
\def\circleC{(-60:5mm) circle (7mm) node[index, below=3mm] (C) {$C$}}

\begin{scope}
    \draw[fill=lightblue] \circleA;
    \draw \circleB;
    \draw \circleC +(0, -12mm) node[index] (label 1) {$A$};
\end{scope}

\node (center 1) at (barycentric cs:A=1,B=1,C=1) {};

\begin{scope}[xshift=28mm]
    \draw[fill=lightblue] \circleB \circleC +(0, -12mm) node[index] (label 2) {$(B\cup C)$};
    \draw \circleA;
\end{scope}

\node (center 2) at (barycentric cs:A=1,B=1,C=1) {};

\begin{scope}[xshift=56mm]
    \begin{scope}\clip \circleB;
        \draw[fill=lightblue] \circleA;
    \end{scope}
    \begin{scope}
        \clip \circleC;
        \draw[fill=lightblue] \circleA;
    \end{scope}
    \draw \circleA \circleB \circleC +(0, -12mm) node[index] (label 3) {$A\cap(B\cup C)$};
\end{scope}

\node (center 3) at (barycentric cs:A=1,B=1,C=1) {};

\begin{scope}[xshift=84mm]
    \begin{scope}
        \clip \circleB;
        \draw[fill=lightblue] \circleA;
    \end{scope}
    \draw \circleA \circleB \circleC +(0, -12mm) node[index] (label 4) {$(A\cap B)$};
\end{scope}

\node (center 4) at (barycentric cs:A=1,B=1,C=1) {};

\begin{scope}[xshift=112mm]
    \begin{scope}
        \clip \circleC;
        \draw[fill=lightblue] \circleA;
    \end{scope}
    \draw \circleA \circleB \circleC +(0, -12mm) node[index] (label 5) {$(A\cap C)$};
\end{scope}

\node (center 5) at (barycentric cs:A=1,B=1,C=1) {};

\path (center 1) -- (center 2) node[index, midway] (midlabel 1) {$\cap$}
      (center 2) -- (center 3) node[index, midway] (midlabel 2) {$=$}
      (center 3) -- (center 4) node[index, midway] (midlabel 3) {$=$}
      (center 4) -- (center 5) node[index, midway] (midlabel 4) {$\cup$};

\path (label 1) -| (midlabel 1) node[index, midway] {$\cap$}
      (label 2) -| (midlabel 2) node[index, midway] {$=$}
      (label 3) -| (midlabel 3) node[index, midway] {$=$}
      (label 4) -| (midlabel 4) node[index, midway] {$\cup$};

\end{tikzpicture}

\begin{tikzpicture}
    \node {17}
        child {node {13}
            child {node {8}
                child {node[discarded, label=left:{$i$}] {20} edge from parent[draw=none]}
                child {node[discarded] {25} edge from parent[draw=none]}
            }
            child {node {7}}
        }
        child {node {5}
            child {node {4}}
            child {node {2}
                child[missing]
                child {node[draw=none, fill=none, label=right:{}] {} edge from parent[draw=none]}
            }
        };
\end{tikzpicture}
}
    \subcaptionbox{\label{fig:6.3-1d}}[0.5\textwidth]{\begin{tikzpicture}

\def\circleA{(0:0) circle (7mm) node[index, above left=1mm and 2mm] (A) {$A$}}
\def\circleB{(0:5mm) circle (7mm) node[index, above right=1mm and 2mm] (B) {$B$}}
\def\circleC{(-60:5mm) circle (7mm) node[index, below=3mm] (C) {$C$}}

\begin{scope}
    \draw[fill=lightblue] \circleA;
    \draw \circleB;
    \draw \circleC +(0, -12mm) node[index] (label 1) {$A$};
\end{scope}

\node (center 1) at (barycentric cs:A=1,B=1,C=1) {};

\begin{scope}[xshift=28mm]
    \draw[fill=lightblue] \circleB \circleC +(0, -12mm) node[index] (label 2) {$(B\cup C)$};
    \draw \circleA;
\end{scope}

\node (center 2) at (barycentric cs:A=1,B=1,C=1) {};

\begin{scope}[xshift=56mm]
    \begin{scope}\clip \circleB;
        \draw[fill=lightblue] \circleA;
    \end{scope}
    \begin{scope}
        \clip \circleC;
        \draw[fill=lightblue] \circleA;
    \end{scope}
    \draw \circleA \circleB \circleC +(0, -12mm) node[index] (label 3) {$A\cap(B\cup C)$};
\end{scope}

\node (center 3) at (barycentric cs:A=1,B=1,C=1) {};

\begin{scope}[xshift=84mm]
    \begin{scope}
        \clip \circleB;
        \draw[fill=lightblue] \circleA;
    \end{scope}
    \draw \circleA \circleB \circleC +(0, -12mm) node[index] (label 4) {$(A\cap B)$};
\end{scope}

\node (center 4) at (barycentric cs:A=1,B=1,C=1) {};

\begin{scope}[xshift=112mm]
    \begin{scope}
        \clip \circleC;
        \draw[fill=lightblue] \circleA;
    \end{scope}
    \draw \circleA \circleB \circleC +(0, -12mm) node[index] (label 5) {$(A\cap C)$};
\end{scope}

\node (center 5) at (barycentric cs:A=1,B=1,C=1) {};

\path (center 1) -- (center 2) node[index, midway] (midlabel 1) {$\cap$}
      (center 2) -- (center 3) node[index, midway] (midlabel 2) {$=$}
      (center 3) -- (center 4) node[index, midway] (midlabel 3) {$=$}
      (center 4) -- (center 5) node[index, midway] (midlabel 4) {$\cup$};

\path (label 1) -| (midlabel 1) node[index, midway] {$\cap$}
      (label 2) -| (midlabel 2) node[index, midway] {$=$}
      (label 3) -| (midlabel 3) node[index, midway] {$=$}
      (label 4) -| (midlabel 4) node[index, midway] {$\cup$};

\end{tikzpicture}

\begin{tikzpicture}[
    level/.append style = {sibling distance=18\vertexsize/3^#1},
    total node/.append style={text width=8mm}
]

\node (root) {1}
    child {node (0) {1}
        child {node (00) {1}
            child {node (000) {}}
            child {node {}}
            child {node {}}
        }
        child {node (01) {1}
            child {node {}}
            child {node {}}
            child {node {}}
        }
        child {node (02) {1}
            child {node {}}
            child {node {}}
            child {node {}}
        }
    }
    child {node (1) {1}
        child {node (10) {1}
            child {node {}}
            child {node {}}
            child {node {}}
        }
        child {node (11) {1}
            child {node {}}
            child {node {}}
            child {node {}}
        }
        child {node (12) {1}
            child {node {}}
            child {node {}}
            child {node {}}
        }
    }
    child {node (2) {1}
        child {node (20) {1}
            child {node {}}
            child {node {}}
            child {node {}}
        }
        child {node (21) {1}
            child {node {}}
            child {node {}}
            child {node {}}
        }
        child {node (22) {1}
            child {node {}}
            child {node {}}
            child {node (222) {}}
        }
    };

\node[below=5mm of 000] (leaf 0) {$\Theta(1)$};
\foreach[count=\i] \x in {0, ..., 5} {
    \node[right=0pt of leaf \x] (leaf \i) {$\Theta(1)$};
}
\node[below=5mm of 222] (leaf m) {$\Theta(1)$};
\node[left=0pt of leaf m] (leaf m minus 1) {$\Theta(1)$};
\path (leaf 6) -- (leaf m minus 1) node[midway, font=\bfseries] {$\dots$};
\foreach \x in {0, ..., 6, m minus 1, m} {
    \draw[transition edge={draw=none}{very densely dashed}{0.5}] (leaf \x |- 000) -- (leaf \x.north);
}

\node[total node, right=of leaf m] (level h total) {$\Theta(3^n)$};
\node[total node] at (level h total |- 22) (level 2 total) {$3^2$};
\node[total node] at (level h total |- 2) (level 1 total) {$3$};
\node[total node] at (level h total |- root) (level 0 total) {$1$};

\path (level 2 total.east) -- (level h total.east) node[midway, xshift=-5mm, font=\bfseries] {$\vdots$};

\draw[level arrow, shorten <=5mm] (root) -- (level 0 total);
\draw[level arrow, shorten <=5mm] (2) -- (level 1 total);
\draw[level arrow, shorten <=5mm] (22) -- (level 2 total);
\draw[level arrow, shorten <=2mm] (leaf m) -- (level h total);

\draw[orange, decorate, decoration={brace, amplitude=8pt, mirror}]
    (leaf 0.south west) -- (leaf m.south east) node[midway, yshift=-6mm, text=black] (leaves total) {$3^n$};

\coordinate (total line end) at (leaves total -| level h total.east);
\draw (2 |- total line end) -- (total line end) node[at end, anchor=east, yshift=-4mm] {Total: $O(3^n)$};

\coordinate[left=4mm of leaf 0] (bottom);
\draw[<->, orange] (bottom) -- (bottom |- root) node[midway, fill=white, text=black] {$n$};

\end{tikzpicture}
}
    \par\vspace{7mm}
    \subcaptionbox{\label{fig:6.3-1e}}[0.5\textwidth]{\begin{tikzpicture}

\def\circleA{(0:0) circle (7mm) node[index, above left=1mm and 2mm] (A) {$A$}}
\def\circleB{(0:5mm) circle (7mm) node[index, above right=1mm and 2mm] (B) {$B$}}
\def\circleC{(-60:5mm) circle (7mm) node[index, below=3mm] (C) {$C$}}

\begin{scope}
    \draw[fill=lightblue] \circleA;
    \draw \circleB;
    \draw \circleC +(0, -12mm) node[index] (label 1) {$A$};
\end{scope}

\node (center 1) at (barycentric cs:A=1,B=1,C=1) {};

\begin{scope}[xshift=28mm]
    \draw[fill=lightblue] \circleB \circleC +(0, -12mm) node[index] (label 2) {$(B\cup C)$};
    \draw \circleA;
\end{scope}

\node (center 2) at (barycentric cs:A=1,B=1,C=1) {};

\begin{scope}[xshift=56mm]
    \begin{scope}\clip \circleB;
        \draw[fill=lightblue] \circleA;
    \end{scope}
    \begin{scope}
        \clip \circleC;
        \draw[fill=lightblue] \circleA;
    \end{scope}
    \draw \circleA \circleB \circleC +(0, -12mm) node[index] (label 3) {$A\cap(B\cup C)$};
\end{scope}

\node (center 3) at (barycentric cs:A=1,B=1,C=1) {};

\begin{scope}[xshift=84mm]
    \begin{scope}
        \clip \circleB;
        \draw[fill=lightblue] \circleA;
    \end{scope}
    \draw \circleA \circleB \circleC +(0, -12mm) node[index] (label 4) {$(A\cap B)$};
\end{scope}

\node (center 4) at (barycentric cs:A=1,B=1,C=1) {};

\begin{scope}[xshift=112mm]
    \begin{scope}
        \clip \circleC;
        \draw[fill=lightblue] \circleA;
    \end{scope}
    \draw \circleA \circleB \circleC +(0, -12mm) node[index] (label 5) {$(A\cap C)$};
\end{scope}

\node (center 5) at (barycentric cs:A=1,B=1,C=1) {};

\path (center 1) -- (center 2) node[index, midway] (midlabel 1) {$\cap$}
      (center 2) -- (center 3) node[index, midway] (midlabel 2) {$=$}
      (center 3) -- (center 4) node[index, midway] (midlabel 3) {$=$}
      (center 4) -- (center 5) node[index, midway] (midlabel 4) {$\cup$};

\path (label 1) -| (midlabel 1) node[index, midway] {$\cap$}
      (label 2) -| (midlabel 2) node[index, midway] {$=$}
      (label 3) -| (midlabel 3) node[index, midway] {$=$}
      (label 4) -| (midlabel 4) node[index, midway] {$\cup$};

\end{tikzpicture}

\begin{tikzpicture}
    \node {8}
        child {node {7}
            child {node {2}
                child {node[discarded] {20} edge from parent[draw=none]}
                child {node[discarded] {25} edge from parent[draw=none]}
            }
            child {node {4}}
        }
        child {node {5}
            child {node[discarded, label=left:{$i$}] {13} edge from parent[draw=none]}
            child {node[discarded] {17} edge from parent[draw=none]
                child[missing]
                child {node[draw=none, fill=none] {} edge from parent[draw=none]}
            }
        };
\end{tikzpicture}
}
    \caption{The operation of \proc{Build-Max-Heap} on the array $A=\langle$5, 3, 17, 10, 84, 19, 6, 22, 9$\rangle$.\,
    \textbf{(a)}\, The array $A$ and the binary tree it represents before the first call to \proc{Max-Heapify} in line 3.\,
    \textbf{(b)--(d)}\, The data structure before each subsequent call to \proc{Max-Heapify}.\,
    \textbf{(e)}\, The resulting max-heap after \proc{Build-Max-Heap} finishes.} \label{fig:6.3-1}
\end{figure}
